\begin{frame}[fragile]
\begin{block}{Analysis}
This is my opinion of Trees That Grow
\end{block}
\end{frame}

\begin{frame}[fragile]
\begin{block}{Alternatives}
We already know that classy lenses (and prisms) work toward resolving this issue
\end{block}
\end{frame}

\begin{frame}[fragile]
\begin{block}{Alternatives}
\begin{lstlisting}[style=haskell]
class HasImage a where
  image :: Lens a Image

instance HasImage ICE_Image where
\end{lstlisting}
\end{block}
\end{frame}

\begin{frame}[fragile]
\begin{block}{Alternatives}
This requires creating a new data type:
\begin{lstlisting}[style=haskell]

data ICE_Image =
  ICE_Image
    ICE
    Image
\end{lstlisting}
\end{block}
\end{frame}

\begin{frame}[fragile]
\begin{block}{One problem}
One problem that I found with TTG is that the type-variable bubbles up the data type tree
\end{block}
\end{frame}

\begin{frame}[fragile]
\begin{block}{One problem}
My \lstinline{Propulsion} data type has 18 type-variables
\end{block}
\end{frame}

\begin{frame}[fragile]
\begin{block}{One problem}
I start running out of names at 26

I didn't get to \lstinline{Aircraft}, which itself, not at the top
\end{block}
\end{frame}

\begin{frame}[fragile]
\begin{block}{Therefore}
I have come to prefer classy lenses and prisms.
\end{block}
\end{frame}

\begin{frame}[fragile]
\begin{block}{Interesting Note}
GHC plans to utilise TTG for its syntax tree, to achieve extensibility.
\end{block}
\end{frame}
